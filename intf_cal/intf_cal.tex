\documentclass[12pt]{article}
\usepackage{graphicx}
\usepackage{placeins}

\input{crisdefs.tex}

\title{A Note on Interferometric Calibration \\
\vspace{3mm}
{****} DRAFT {****} \\
}

\author{H.~E.~Motteler and L.~L.~Strow \\
  \\
  UMBC Atmospheric Spectroscopy Lab \\
  Joint Center for Earth Systems Technology \\
}

\date{\today}
\begin{document}

\maketitle

We derive the UW definition of reference truth convolved with
responsivity.  Note that this does not prove the definition is
``correct'' in any particular way.  The definition is just an
approximation to a more accurate representation.  It may suggest why
the UW definition is easier to use as a specification for some forms
of the calibration equation.

\vspace{2mm}
Calibration of the on-axis optical path of a Michaelson
interferometer can be represented as

\begin{equation}\label{cal:1}
  \rcal = \rIT  \frac{\ES - \SP}{\IT - \SP}
\end{equation} 

\noindent
where $\rcal$ is calibrated radiances, $\rIT$ is expected radiance
from the internal calibration target, and $\ES$, $\IT$, and $\SP$
are uncalibrated spectra for earth scene, internal calibration
target, and space looks, respectively.  We can approximate
instrument responsivity as

\begin{equation}\label{resp}
  \rho = \frac{\IT - \SP}{\rIT}
\end{equation} 

\noindent
Substituting (\ref{resp}) in (\ref{cal:1}) gives

\begin{equation}\label{cal:2}
  \rcal =\frac{\ES - \SP}{\rho}
\end{equation} 

We can represent $\ES$, $\IT$, and $\SP$ as $\ES = F \rho \, \rES$,
$\IT = F \rho \, \rIT$, and $\SP= F \rho \, \rSP$, where $\rES$,
$\rIT$, and $\rSP$ are high resolution approximations to the true
radiances, $F$ is resampling from the high resolution to $dv = 1/
(2\, \opd)$, and $\opd$ is the optical path difference.
Substituting this into (\ref{cal:2}) gives

\begin{equation}\label{cal:3}
  \rcal = \frac{F\rho \, \rES - F\rho \, \rSP}{\rho}
\end{equation} 

\noindent
The space look radiances are very small in comparison with earth
scene radiances.  If we drop the space look term in (\ref{cal:3}) 
we have

\begin{equation}
  \rcal \approx \rresp = \frac{F\rho \, \rES}{\rho}
\end{equation} 

\noindent
This is the UW definition of ``reference truth with responsivity''.

A more conventional and user-friendly definition of reference truth
is
\begin{equation}
  \rflat =  F \rES
\end{equation} 

\noindent
In practice we find the \umbc\ \ccast\ reference calibration
equation (a ``ratio first'' form) has smaller residuals when
compared with $\rflat$, while the \noaa\ 4 algorithm (an
``$\SA^{-1}$ first'' form) has smaller residuals with $\rresp$.  
It seems to us the proper focus for calibration algorithm
development should be minimizing residuals in comparison with
$\rflat$.


\end{document}

