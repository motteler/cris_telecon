\documentclass[12pt]{article}
\usepackage{graphicx}
\usepackage{placeins}
\usepackage{amsmath}

\input{crisdefs.tex}


\title{A Note on Interferometric Calibration \\
\vspace{3mm}
{****} DRAFT {****} \\
}

\author{H.~E.~Motteler and L.~L.~Strow \\
  \\
  UMBC Atmospheric Spectroscopy Lab \\
  Joint Center for Earth Systems Technology \\
}

\date{\today}
\begin{document}

\maketitle

We derive the UW definition of reference truth convolved with
responsivity.  Note that this does not prove the definition is
``correct'' in any particular way, just that it follows from the
assumption that responsivity does not commute with resampling for
earth-scene looks, that is, that $F\rho\,\rES \ne \rho\,F\,\rES$.
where $\rES$ is a high resolution approximation to earth-scene
radiance, $\rho$ is instrument responsivity, and $F$ is resampling
from the high resolution to the sensor grid.  The derivation may
suggest why the UW definition works better as a specification for
some forms of the calibration equation.

\vspace{2mm}
Calibration of the on-axis optical path of a Michaelson
interferometer can be represented as

\begin{equation}\label{caldef}
  \rcalLR = \rITLR  \frac{\ES - \SP}{\IT - \SP}
\end{equation} 

\noindent
where $\rcalLR$ is calibrated radiances, $\rITLR$ is expected
radiance from the internal calibration target, and $\ES$, $\IT$, and
$\SP$ are uncalibrated spectra for earth scene, internal calibration
target, and space looks, respectively.  All values in equation
\ref{caldef} are at the ``sensor grid'' $dv = 1/ (2\, \opd)$, where
$\opd$ is the optical path difference.

Let $\ES \approx F\rho(\rES+\rSP)$, $\IT \approx F\rho(\rIT+\rSP)$,
and $\SP \approx F\rho\,\rSP$, where $\rES$, $\rIT$, and $\rSP$ are
high resolution approximations to the true radiances, $\rho$ is
responsivity, and $F$ is resampling from the high resolution to the
sensor grid $dv$.  Let $\rITLR = F(\rIT)$.  Substituting this into
(\ref{caldef}) gives

\begin{align}
  \rcalLR &\approx \rITLR \frac{F\rho(\rES+\rSP) - F\rho\,\rSP}
                               {F\rho(\rIT+\rSP) - F\rho\,\rSP} \notag \\
%         &= \rITLR \frac{F\rho\,\rES + F\rho\,\rSP - F\rho\,\rSP}
%                        {F\rho\,\rIT + F\rho\,\rSP - F\rho\,\rSP} \\
          &= \rITLR \frac{F\rho\,\rES}
                         {F\rho\,\rIT} \label{cal1} \\
          &= \rITLR \frac{F\rho\,\rES}
                         {\rho\,F\,\rIT} \label{cal2} \\
          &= \frac{F\rho\,\rES} 
                  {\rho} = \rresp \label{calUW}
\end{align} 

\noindent
We go from equation \ref{cal1} to \ref{cal2} because responsivity
commutes with resampling for the ICT look, that is, $F\rho\,\rIT =
\rho\,F\,\rIT$.  The key assumption underlying the UW definition is
that responsivity does {\em not} commute with resampling for the ES
look, that is, that $F\rho\,\rES \ne \rho\,F\,\rES$.  Equation
\ref{calUW} is the UW definition of ``reference truth with
responsivity''.

\vspace{2mm}
A more conventional and user-friendly definition of reference truth
is
\begin{equation}
  \rflat =  F \rES
\end{equation} 

\noindent
In practice we find the ``ratio first'' \umbc\ \ccast\ reference
calibration equation has smaller residuals when compared with
$\rflat$, while the ``$\SA^{-1}$ first'' \noaa~4 algorithm has
smaller residuals with $\rresp$.  It seems to us the proper focus
for calibration algorithm development should be minimizing residuals
in comparison with $\rflat$.


\end{document}

