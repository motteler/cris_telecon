\documentclass[12pt]{article}
\usepackage{graphicx}
\usepackage{placeins}
\usepackage{amsmath}

\input{crisdefs.tex}


\title{A Note on Interferometric Calibration \\
\vspace{3mm}
{****} DRAFT {****} \\
}

\author{H.~E.~Motteler and L.~L.~Strow \\
  \\
  UMBC Atmospheric Spectroscopy Lab \\
  Joint Center for Earth Systems Technology \\
}

\date{\today}
\begin{document}

\maketitle

We consider the definition of reference truth for measurements made
with a Michelson interferometer, taking into account relatively
small (or possibly non-existant) effects such as filter position and
potential mathematical artifacts.  The immediate application is
defining reference truth and a corresponding calibration equation
for the for the {\cris} instrument.

% We want to be careful with definitions to separate mathematical
% and physical artifacts in subsequent derivations.

% pedantic because looking for small effects

\section{Michelson Interferometer}

\begin{figure}
\begin{center}
  \includegraphics[scale=0.5]{figures/mich_filt2.pdf}
\caption{Michelson interferometer with filters}
\label{intf1}
\end{center}
\end{figure}

Figure \ref{intf1} shows a basic Michelson interferometer.  Let
$\rin(\nu)$ be incoming radiance as a function of frequency $\nu$,
$\delta$ mirror displacement, and $\rout(\nu, \delta)$ radiance on
the path to the detector.  In practice the signal from the detector
is the product of incoming radiance, beam-splitter efficiency, and
detector responsivity.  But suppose for the moment that we have a
perfect beam splitter and mirrors, and that there are no filters.
Then radiance $\rout$ on the path to the detector can be represented
as

\begin{equation}\label{eq1}
  \rout(\nu,\delta) = \rin(\nu)(1+\cos 2\pi\nu\delta)/2
\end{equation}

\noindent
including a term $\rin(\nu)/2$ that depends only on $\nu$.
Integrating over frequency, we have

\begin{equation}\label{eq2}
  \rout(\delta) = \frac{1}{2}\int_{\nu=0}^{\inf}\!\rin(\nu)d\nu + 
     \frac{1}{2}\int_{\nu=0}^{\inf}\!\rin(\nu)\cos(2\pi\nu\delta)d\nu
\end{equation}
\noindent
This is the continuous interferogram as a function of
displacement~$\delta$.

We are interested in the following question: if we have a single
filter $f(\nu)$, does it make a difference if $f$ is on the input 
or detector leg of the interferometer?  We are interested in the
following question: if we have a single filter $f(\nu)$, does
$\rout$ (output radiance, in our diagram) change when $f$ is moved
from the input to the detector leg of the interferometer?
We present two related arguments that it does not.  Consider figure
\ref{intf1} again, and suppose we have $f = f_1(\nu)$ on the input
leg and nothing on the output leg.  This can be represented as

\begin{equation}\label{eq3}
  \rout(\nu,\delta) = f(\nu)\rin(\nu)(1+\cos 2\pi\nu\delta)/2
\end{equation}

Now suppose we move $f$ to the detector leg.  The assumption in the
literature seems to be that equation (\ref{eq3}) stll holds.  Note
that filter position does not matter for the $\rin(\nu)/2$ term, or
for the case when we remove one of the mirrors.

Consider two lines at frequencies $\nu_1$ and $\nu_2$, where $\nu_1$
is in the passband of $f$ and $\nu_2$ is not.  If $f$ is on the
detector leg then both $\nu_1$ and $\nu_2$ will participate in
interference, with radiance before the filter as in equation
(\ref{eq1}).  After the filter we have $\rout(\nu_2,\delta) = 0$ for
all $\delta$, with $\rout(\nu_1,\delta)$ unchanged.  If $f$ is on
the input leg then $\nu_1$ will be as before but $\nu_2$ will not be
present anywhere downstream, so again we have $\rout(\nu_2,\delta) =
0$ for all $\delta$.  So the cases are the same.

Note that in this hypothetical situation there is no nonlinearity or
intermodulation because we are assuming a perfect beamsplitter and
mirrors, and no Nyquest limit, truncation, or discretization errors
because we are considering the case for arbitraly real-valued
$\delta$.  But the argument may not apply as $\nu_1 \rightarrow
\nu_2$, because at some point $f$ can not separate them.

The case we are interested in practice is where $\nu_1$ and $\nu_2$
are close and fall on a slope or shoulder of $f$.  But for the
following argument we don't need to assume that.  So consider the
case of arbitrary $\nu_1$ and $\nu_2$.  The AC component of equation
(\ref{eq2}) with $f$ on the input leg is

\begin{equation}\label{eq4}
  \rout(\delta) = \int_{\nu=0}^{\inf}f(\nu)\cos(2\pi\nu\delta)d\nu
\end{equation}

\noindent
For the case where $\nu$ takes on only the two values $\nu_1$ and
$\nu_2$ this simplifies to

\begin{equation}\label{eq5}
  \rout(\delta) = f(\nu_1)\cos(2\pi\nu_1\delta) + 
                  f(\nu_2)\cos(2\pi\nu_2\delta)
\end{equation}

\noindent
This is interesting because of the recognizable beat pattern in 
the interferogram.  Now consider $f$ on the output leg.  On the 
path before $f$ we have the case of no filter,

\begin{equation}\label{eq6}
  \rout(\delta) = \cos(2\pi\nu_1\delta) + 
                  \cos(2\pi\nu_2\delta)
\end{equation}

\noindent
And after $f$ we have equation (\ref{eq5}) again.  Since $\nu_1$ and
$\nu_2$ were chosen arbitrarily we conclude that the property holds
for all $\nu$ and $\delta$, and that the filter position does not
matter.

In practice, $f_1$ might be beam splitter efficiency, $f_2$ detector
responsivity, and $g(\rout(\delta))$ a function taking radiance to
voltage.  In the case of the {\cris} instrument, $f_2$ could be
detector responsivity plus optical filter effects.  The particular
question for {\cris} we were interested in was if it was correct
to model the effects of a filter on the detector leg with a filter
on the input.  The answer seems to be yes.

\end{document}

\FloatBarrier

\section{CrIS reference truth }

We can derive an approximation to the UW definition of reference
truth convolved with responsivity from some plausible assumptions
about filters and resampling.  Suppose $\rES$ is high resolution
earth scene radiance, $\rIT$ high resolution black-body radiance
from the calibration target, $\rho$ instrument responsivity, and $R$
is resampling from the high resolution to the sensor grid.  Then
$R\rho\,\rIT \approx \rho\,R\,\rIT$, for ICT looks, but $R\rho\,\rES
\ne \rho\,R\,\rES$ for earth scenes.  

% The derivation may suggest why the UW definition works better as a
% specification for some forms of the calibration equation.

\vspace{2mm}
Calibration of the on-axis optical path of a Michaelson
interferometer can be represented as

\begin{equation}\label{caldef}
  \rcalLR = \rITLR  \frac{\ES - \SP}{\IT - \SP}
\end{equation} 

\noindent
where $\rcalLR$ is calibrated radiances, $\rITLR$ is expected
radiance from the internal calibration target, and $\ES$, $\IT$, and
$\SP$ are uncalibrated spectra for earth scene, internal calibration
target, and space looks, respectively.  All values in equation
\ref{caldef} are at the ``sensor grid'' $dv = 1/ (2\, \opd)$, where
$\opd$ is the optical path difference.

Let $\ES \approx R\rho(\rES+\rSP)$, $\IT \approx R\rho(\rIT+\rSP)$,
and $\SP \approx R\rho\,\rSP$, where $\rES$, $\rIT$, and $\rSP$ are
high resolution approximations to the true radiances, $\rho$ is
responsivity, and $R$ is resampling from the high resolution to the
sensor grid $dv$.  Let $\rITLR = R(\rIT)$.  Substituting this into
(\ref{caldef}) gives

\begin{align}
  \rcalLR &\approx \rITLR \frac{R\rho(\rES+\rSP) - R\rho\,\rSP}
                               {R\rho(\rIT+\rSP) - R\rho\,\rSP} \notag \\
%         &= \rITLR \frac{R\rho\,\rES + R\rho\,\rSP - R\rho\,\rSP}
%                        {R\rho\,\rIT + R\rho\,\rSP - R\rho\,\rSP} \\
          &= \rITLR \frac{R\rho\,\rES}
                         {R\rho\,\rIT} \label{cal1} \\
          &\approx \rITLR \frac{R\rho\,\rES}
                               {\rho\,R\,\rIT} \label{cal2} \\
          &= \frac{R\rho\,\rES} 
                  {\rho} = \rresp \label{calUW}
\end{align} 

\noindent
We go from equation \ref{cal1} to \ref{cal2} because responsivity
commutes at least approximately with resampling for the ICT look,
that is, $R\rho\,\rIT \approx \rho\,R\,\rIT$, but not as well for
the ES look, that is, that $R\rho\,\rES \ne \rho\,R\,\rES$.
Equation \ref{calUW} is the UW definition of ``reference truth with
responsivity''.

\vspace{2mm}
A more conventional and user-friendly definition of reference truth
is
\begin{equation}
  \rflat =  R \rES
\end{equation} 

\noindent
If we assume $R\rho\,\rES \approx \rho\,R\,\rES$ then we get
``flat'' reference truth.  [show residual figures]

In practice we find the ``ratio first'' \umbc\ \ccast\ reference
calibration equation has smaller residuals when compared with
$\rflat$, while the ``$\SA^{-1}$ first'' \noaa~4 algorithm has
smaller residuals with $\rresp$.  It seems to us the proper focus
for calibration algorithm development should be minimizing residuals
in comparison with $\rflat$.


\end{document}

