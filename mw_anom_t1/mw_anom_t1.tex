\documentclass[10pt]{beamer}
% \usetheme{Boadilla}
  \usetheme{default}

\input{crisdefs.tex}

\title{A preliminary comparison \\
       of CrIS processing algorithms \\
       after the SNPP CrIS MWIR anomaly \\
}
\author{H.~E.~Motteler, L.~L.~Strow}
\institute{
  UMBC Atmospheric Spectroscopy Lab \\
  Joint Center for Earth Systems Technology \\
}
\date{\today}
\begin{document}

%----------- slide --------------------------------------------------%
\begin{frame}[plain]
\titlepage
\end{frame}
%----------- slide --------------------------------------------------%
\begin{frame}
\frametitle{Introduction}
\begin{itemize}

  \item We present a preliminary comparison of NPP CrIS high res
    products for the NOAA A4, UMBC CCAST A4, and UMBC CCAST
    reference calibration algorithms, before and after the NPP MW
    anomaly.

  \item Complex residuals for the NOAA product are slightly larger
    before and significanty larger after the anomaly.

  \item Differences in the NOAA and UMBC CCAST processing include
  \begin{itemize}
    \item CCAST uses cosine apodization of the extended point set
      outside the user-grid OPD, rather than truncation or circular
      shift

    \item CCAST uses ``small n'' psinc (psinc at the decimated
      sensor grid) for both the ILS and SA matrix

    \item CCAST spectral-space processing filters for the reference
      algorithm have a slightly narrower passband and more gradual
      rolloff than the A4 filters.  The difference is greatest at
      the low end of the LW band.

%   \item a preference for the CCAST reference vs the A4 algorithm,
%     primarily because the former works better for reference truth
%     defined without convolution with instrument responsivity
  \end{itemize}

  \item The CCAST reference and NOAA A4 calibration equations and
    CCAST apodization are described in an appendix.

\end{itemize}
\end{frame}
%----------- slide --------------------------------------------------%
\begin{frame}
\frametitle{Test Data}
\begin{itemize}

\item NOAA data are GCRSO-SCRIF HDF5 files from NOAA CLASS.  CCAST
  data are from our regular production runs, with the A4 algorithm
  selectable as an option.

  \item CCAST uses 45-scan and NOAA 60-scan granules.  For these
    preliminary tests we simply took the intersection of scans for
    two overlapping granules, one pair before and one pair after the
    anomaly.

  \item The pre-anomaly test set consists of 1110 obs from
    relatively warm granules, midpoint time 03-Mar-2019 04:07:11.
    The set is divided into even and odd sweeps (even and odd FORs)
    and broken out by FOV.

  \item The post-anomaly test set consists of 1260 obs from another
    pair of warm granules, midpoint time 21-Apr-2019 04:08:29.  The
    set is also divided into even and odd sweeps and broken out by
    FOV.

  \item We show representative cases for corner, side, and center
    FOVs, for both sweep directions.

\end{itemize}
\end{frame}
%----------- slide --------------------------------------------------%
\begin{frame}
\frametitle{LW FOV 1 Sweep 0 BT Spectra}
\begin{columns}[t]
\begin{column}{0.5\textwidth}
  \begin{centering}
  \includegraphics[width=\textwidth]{figures/LW_MW_good_BT_fov1_sd0.pdf}
  \end{centering}\vspace{3mm}
  Pre-anomaly mean BT spectra for NOAA A4, CCAST ref, and CCAST A4.
  The CCAST A4 and NOAA A4 algorithms are in good agreement, aside
  from the slight dip at the low end of the LW.

\end{column}
\begin{column}{0.5\textwidth}  
  \begin{centering}
  \includegraphics[width=\textwidth]{figures/LW_MW_fail_BT_fov1_sd0.pdf}
  \end{centering}\vspace{3mm}
  Post-anomaly mean BT spectra for NOAA A4, CCAST ref, and CCAST A4.
  The NOAA and CCAST algorithms are diverging.

\end{column}
\end{columns}
\end{frame}
%----------- slide --------------------------------------------------%
\begin{frame}
\frametitle{LW FOV 1 Sweep 1 BT Spectra}
\begin{columns}[t]
\begin{column}{0.5\textwidth}
  \begin{centering}
  \includegraphics[width=\textwidth]{figures/LW_MW_good_BT_fov1_sd1.pdf}
  \end{centering}\vspace{3mm}
  Pre-anomaly mean BT spectra for NOAA A4, CCAST ref, and CCAST A4.
  The CCAST A4 and NOAA A4 algorithms are in good agreement, aside
  from the slight dip at the low end of the LW.

\end{column}
\begin{column}{0.5\textwidth}  
  \begin{centering}
  \includegraphics[width=\textwidth]{figures/LW_MW_fail_BT_fov1_sd1.pdf}
  \end{centering}\vspace{3mm}
  Post-anomaly mean BT spectra for NOAA A4, CCAST ref, and CCAST A4.
  The divergence is larger for sweep direction 1.

\end{column}
\end{columns}
\end{frame}
%----------- slide --------------------------------------------------%
\begin{frame}
\frametitle{LW FOV 1 Sweep 0 Complex Residuals}
\begin{columns}[t]
\begin{column}{0.5\textwidth}
  \begin{centering}
  \includegraphics[width=\textwidth]{figures/LW_MW_good_imag_fov1_sd0.pdf}
  \end{centering}\vspace{3mm}
  Pre-anomaly complex residual mean and standard deviation.  The
  NOAA complex residual is a little larger.

\end{column}
\begin{column}{0.5\textwidth}  
  \begin{centering}
  \includegraphics[width=\textwidth]{figures/LW_MW_fail_imag_fov1_sd0.pdf}
  \end{centering}\vspace{3mm}
  Post-anomaly complex residual mean and standard deviation.  \\ The
  NOAA complex residual is significantly larger and the standard
  deviations have diverged.

\end{column}
\end{columns}
\end{frame}
%----------- slide --------------------------------------------------%
\begin{frame}
\frametitle{LW FOV 1 Sweep 1 Complex Residuals}
\begin{columns}[t]
\begin{column}{0.5\textwidth}
  \begin{centering}
  \includegraphics[width=\textwidth]{figures/LW_MW_good_imag_fov1_sd1.pdf}
  \end{centering}\vspace{3mm}
  Pre-anomaly complex residual mean and standard deviation.  The NOAA complex
  residual is a little larger.

\end{column}
\begin{column}{0.5\textwidth}  
  \begin{centering}
  \includegraphics[width=\textwidth]{figures/LW_MW_fail_imag_fov1_sd1.pdf}
  \end{centering}\vspace{3mm}
  Post-anomaly complex residual mean and standard deviation.  \\ The
  NOAA complex residual is significantly larger and the standard
  deviations have diverged.

\end{column}
\end{columns}
\end{frame}
%----------- slide --------------------------------------------------%
\begin{frame}
\frametitle{LW FOV 2 Sweep 0 BT Spectra}
\begin{columns}[t]
\begin{column}{0.5\textwidth}
  \begin{centering}
  \includegraphics[width=\textwidth]{figures/LW_MW_good_BT_fov2_sd0.pdf}
  \end{centering}\vspace{3mm}
  Pre-anomaly mean BT spectra.  \\The CCAST A4 and NOAA A4 algorithms
  are in good agreement.

\end{column}
\begin{column}{0.5\textwidth}  
  \begin{centering}
  \includegraphics[width=\textwidth]{figures/LW_MW_fail_BT_fov2_sd0.pdf}
  \end{centering}\vspace{3mm}
  Post-anomaly mean BT spectra.  The NOAA and CCAST algorithms are
  diverging.

\end{column}
\end{columns}

\end{frame}
%----------- slide --------------------------------------------------%
\begin{frame}
\frametitle{LW FOV 2 Sweep 1 BT Spectra}
\begin{columns}[t]
\begin{column}{0.5\textwidth}
  \begin{centering}
  \includegraphics[width=\textwidth]{figures/LW_MW_good_BT_fov2_sd1.pdf}
  \end{centering}\vspace{3mm}
  Pre-anomaly mean BT spectra.  \\The CCAST A4 and NOAA A4 algorithms
  are in good agreement.

\end{column}
\begin{column}{0.5\textwidth}  
  \begin{centering}
  \includegraphics[width=\textwidth]{figures/LW_MW_fail_BT_fov2_sd1.pdf}
  \end{centering}\vspace{3mm}
  Post-anomaly mean BT spectra.  The NOAA and CCAST algorithms are
  diverging.

\end{column}
\end{columns}
\end{frame}
%----------- slide --------------------------------------------------%
\begin{frame}
\frametitle{LW FOV 2 Sweep 0 Complex Residuals}
\begin{columns}[t]
\begin{column}{0.5\textwidth}
  \begin{centering}
  \includegraphics[width=\textwidth]{figures/LW_MW_good_imag_fov2_sd0.pdf}
  \end{centering}\vspace{3mm}
  Pre-anomaly complex residual \\ mean and standard deviation.  \\The
  NOAA complex residual is significantly larger even before the
  anomaly.
  

\end{column}
\begin{column}{0.5\textwidth}  
  \begin{centering}
  \includegraphics[width=\textwidth]{figures/LW_MW_fail_imag_fov2_sd0.pdf}
  \end{centering}\vspace{3mm}
  Post-anomaly complex residual mean and standard deviation.  \\The
  NOAA complex residual is significantly larger and the standard
  deviations have diverged.

\end{column}
\end{columns}
\end{frame}
%----------- slide --------------------------------------------------%
\begin{frame}
\frametitle{LW FOV 2 Sweep 1 Complex Residuals}
\begin{columns}[t]
\begin{column}{0.5\textwidth}
  \begin{centering}
  \includegraphics[width=\textwidth]{figures/LW_MW_good_imag_fov2_sd1.pdf}
  \end{centering}\vspace{3mm}
  Pre-anomaly complex residual mean and standard deviation.  The NOAA complex
  residual is significantly larger even before the anomaly.

\end{column}
\begin{column}{0.5\textwidth}  
  \begin{centering}
  \includegraphics[width=\textwidth]{figures/LW_MW_fail_imag_fov2_sd1.pdf}
  \end{centering}\vspace{3mm}
  Post-anomaly complex residual mean and standard deviation.  \\The
  NOAA complex residual is significantly larger and the standard
  deviations have diverged.

\end{column}
\end{columns}
\end{frame}
%----------- slide --------------------------------------------------%
\begin{frame}
\frametitle{LW FOV 5 Sweep 0 BT Spectra}
\begin{columns}[t]
\begin{column}{0.5\textwidth}
  \begin{centering}
  \includegraphics[width=\textwidth]{figures/LW_MW_good_BT_fov5_sd0.pdf}
  \end{centering}\vspace{3mm}
  Pre-anomaly mean BT spectra.  \\The CCAST and NOAA algorithms
  are in moderately good agreement.

\end{column}
\begin{column}{0.5\textwidth}  
  \begin{centering}
  \includegraphics[width=\textwidth]{figures/LW_MW_fail_BT_fov5_sd0.pdf}
  \end{centering}\vspace{3mm}
  Post-anomaly mean BT spectra.  The NOAA and CCAST algorithms are
  beginning to diverge.

\end{column}
\end{columns}
\end{frame}
%----------- slide --------------------------------------------------%
\begin{frame}
\frametitle{LW FOV 5 Sweep 1 BT Spectra}
\begin{columns}[t]
\begin{column}{0.5\textwidth}
  \begin{centering}
  \includegraphics[width=\textwidth]{figures/LW_MW_good_BT_fov5_sd1.pdf}
  \end{centering}\vspace{3mm}
  Pre-anomaly mean BT spectra.  \\The CCAST and NOAA algorithms are
  in moderately good agreement.

\end{column}
\begin{column}{0.5\textwidth}  
  \begin{centering}
  \includegraphics[width=\textwidth]{figures/LW_MW_fail_BT_fov5_sd1.pdf}
  \end{centering}\vspace{3mm}
  Post-anomaly mean BT spectra.  The NOAA and CCAST algorithms are
  diverging.

\end{column}
\end{columns}
\end{frame}
%----------- slide --------------------------------------------------%
\begin{frame}
\frametitle{LW FOV 5 Sweep 0 Complex Residuals}
\begin{columns}[t]
\begin{column}{0.5\textwidth}
  \begin{centering}
  \includegraphics[width=\textwidth]{figures/LW_MW_good_imag_fov5_sd0.pdf}
  \end{centering}\vspace{3mm}
  Pre-anomaly complex residual mean and standard deviation.  The NOAA complex
  residual is a little larger.

\end{column}
\begin{column}{0.5\textwidth}  
  \begin{centering}
  \includegraphics[width=\textwidth]{figures/LW_MW_fail_imag_fov5_sd0.pdf}
  \end{centering}\vspace{3mm}
  Post-anomaly complex residual mean and standard deviation.  Both residuals are
  larger.

\end{column}
\end{columns}
\end{frame}
%----------- slide --------------------------------------------------%
\begin{frame}
\frametitle{LW FOV 5 Sweep 1 Complex Residuals}
\begin{columns}[t]
\begin{column}{0.5\textwidth}
  \begin{centering}
  \includegraphics[width=\textwidth]{figures/LW_MW_good_imag_fov5_sd1.pdf}
  \end{centering}\vspace{3mm}
  Pre-anomaly complex residual mean and standard deviation.  The NOAA complex
  residual is a little larger.

\end{column}
\begin{column}{0.5\textwidth}  
  \begin{centering}
  \includegraphics[width=\textwidth]{figures/LW_MW_fail_imag_fov5_sd1.pdf}
  \end{centering}\vspace{3mm}
  Post-anomaly complex residual mean and standard deviation.  The
  NOAA residual is larger and the standard deviations have diverged.

\end{column}
\end{columns}
\end{frame}
%----------- slide --------------------------------------------------%
\begin{frame}
\frametitle{SW FOV 2 Sweep 1 BT Spectra}
\begin{columns}[t]
\begin{column}{0.5\textwidth}
  \begin{centering}
  \includegraphics[width=\textwidth]{figures/SW_MW_good_BT_fov2_sd1.pdf}
  \end{centering}\vspace{3mm}
  Pre-anomaly mean BT spectra.  The CCAST and NOAA algorithms are in
  relatively good agreement (but note the extended x axis vs the LW
  plots).

\end{column}
\begin{column}{0.5\textwidth}  
  \begin{centering}
  \includegraphics[width=\textwidth]{figures/SW_MW_fail_BT_fov2_sd1.pdf}
  \end{centering}\vspace{3mm}
  Post-anomaly mean BT spectra.  The NOAA and CCAST algorithms are
  diverging.

\end{column}
\end{columns}
\end{frame}
%----------- slide --------------------------------------------------%
\begin{frame}
\frametitle{SW FOV 2 Sweep 1 Complex Residuals}
\begin{columns}[t]
\begin{column}{0.5\textwidth}
  \begin{centering}
  \includegraphics[width=\textwidth]{figures/SW_MW_good_imag_fov2_sd1.pdf}
  \end{centering}\vspace{3mm}
  Pre-anomaly complex residual \\mean and standard deviation.  \\The
  complex residuals are similar, but the ccast standard deviation is
  significantly larger.

\end{column}
\begin{column}{0.5\textwidth}  
  \begin{centering}
  \includegraphics[width=\textwidth]{figures/SW_MW_fail_imag_fov2_sd1.pdf}
  \end{centering}\vspace{3mm}
  Post-anomaly complex residual mean and standard deviation.  The
  NOAA complex residual is significantly larger at the low end of
  the band.

\end{column}
\end{columns}
\end{frame}
%----------- slide --------------------------------------------------%
\begin{frame}
\frametitle{SW FOV 5 Sweep 0 Complex Residuals}
\begin{columns}[t]
\begin{column}{0.5\textwidth}
  \begin{centering}
  \includegraphics[width=\textwidth]{figures/SW_MW_good_imag_fov5_sd0.pdf}
  \end{centering}\vspace{3mm}
  Pre-anomaly complex residual \\mean and standard deviation.  \\The
  complex residuals and standard deviations are similar.

\end{column}
\begin{column}{0.5\textwidth}  
  \begin{centering}
  \includegraphics[width=\textwidth]{figures/SW_MW_fail_imag_fov5_sd0.pdf}
  \end{centering}\vspace{3mm}
  Post-anomaly complex residual mean and standard deviation.  \\The
  NOAA complex residual is significantly larger.

\end{column}
\end{columns}
\end{frame}
%----------- slide --------------------------------------------------%
\begin{frame}
\frametitle{Conclusions}
\begin{itemize}

  \item Complex residuals for the NOAA product are slightly larger
    before and significanty larger after the anomaly.

  \item This may be due to the mild cosine apodization working
    better for the case of a significant ZPD shift.

  \item Because the complex residuals are smaller, we suspect
    the CCAST BT spectra are closer to reference truth after the
    anomaly.  But this needs to be verified.

  \item The slightly larger NOAA complex residuals before the
    anomaly are puzzling, we do not recall seeing this in earlier
    tests.

  \item The CCAST complex residual standard deviation is larger
    than NOAA for the SW side and corner FOVs.  This may be due to
    the different forms of the SA matrix.

  \item The next step is comparison with calculated reference truth,
    from clear matchups.

\end{itemize}
\end{frame}
%----------- slide --------------------------------------------------%
\begin{frame}
\frametitle{CCAST Calibration Equations}

% \begin{equation}
%   \rES = F \cdot \rIT \cdot f \cdot \SA^{-1}\cdot f \cdot 
%          \frac{\ES - \SPmean}{\ITmean - \SPmean}
% \end{equation}

\begin{center}
CCAST reference equation
\end{center}
% \vspace{-2mm}
\begin{equation*}
  \rESuser = F \cdot \rITsensor \cdot \fcos \cdot \SA^{-1}\cdot \fcos 
         \cdot \frac{\Delta\ES}{\Delta\IT}
\end{equation*}

\vspace{5mm}
\begin{center}
NOAA algorithm A4
\end{center}
% \vspace{-2mm}
\begin{equation*}
  \rESuser = \rITuser \cdot
         \frac{F \cdot \fatbd \cdot \SA^{-1}\cdot \fatbd \cdot 
                  (\frac{\Delta\ES}{\Delta\IT}) \cdot {|\Delta\IT|}}
              {F \cdot \fatbd \cdot \SA^{-1}\cdot \fatbd \cdot {|\Delta\IT|}}
\end{equation*}

\vspace{8mm}
\[ \Delta\ES = \ES - \SPmean, \hspace{4mm}
   \Delta\IT = \ITmean - \SPmean \]

\end{frame}
%----------- slide --------------------------------------------------%
\begin{frame}
\frametitle{Extended Interferogram Apodization}

\begin{itemize}
  \item we can drop 6 points on each side of the extended resolution
    interferograms and stay very close to the {\opd} spec

 \item $(n - 12) \cdot dx =$ 1.5995 for the LW, 1.6081 for the MW,
   and 1.6001 for the SW bands, for typical metrology laser values

 \item this includes the MW band, even though that was not extended
   in the recent update

 \item apodizing (rather than dropping) these points leaves
   effective resolution within specs

 \item note that the apodized extension discussed here is distinct
   from any downstream apodization (such as Hamming) that is applied
   to user-grid radiances

\end{itemize}

\end{frame}
%----------- slide --------------------------------------------------%
\begin{frame}
\frametitle{Extended Interferogram Apodization}
\begin{center}
  \includegraphics[scale=0.5]{figures/apod_LW.pdf}
\end{center}
edge of band details for the LW extended resolution apodization.
The apodization is symmetrical and all the weights are non-zero.
\end{frame}
%----------- slide --------------------------------------------------%

\end{document}

